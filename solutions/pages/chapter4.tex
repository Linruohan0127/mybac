
\chapter{视频动作捕捉动画制作}
\echapter{Preface}

动作捕捉(MoCap, Motion Capture)是指利用多种传感器,如光感、惯性、摄像头等,来记录物体或人体运动过程中的动作的技术。在军事、影视、运动、医疗、机器人、计算机视觉及图形学等领域有着广泛且深入的应用。本章针对动作捕捉在电影制作、游戏开发等需要将动作绑定至角色模型制作三维动画的场景,利用深度学习人体姿态估计,对RGB单目单人视频中的人体动作进行提取,再绑定至角色模型,搭建低门槛、低成本、广泛场景的视频动作捕捉解决方案。

\section{整体框架}

视频动作捕捉动画制作的整体流程如图x所示,首先利用RGB摄像头,如手机摄像头,采集单人运动视频。第一步将视频送入第三章中的二维人体姿态估计模型对每一帧中的主要人物目标的各关节坐标进行估计。第二步将得到的二维坐标序列送入第四章的三维人体姿态估计模型进行三维坐标的估计。第三步,搭建人体骨骼模型,将三维坐标序列转换为描述骨骼运动的欧拉角序列,并写为bvh动作描述文件。第四步创建人体角色模型,利用动画制作软件将bvh文件描述的动作序列绑定至模型。最终生成三维动画,可在动画引擎中进行播放。

\section{实现细节}
\subsection{Bvh文件}{}
BVH(Biovision hierarchical data)是BioVision等动作捕捉设备对人体运动进行捕获后产生文件格式的文件扩展名。BVH文件主要记录了角色的骨骼树和其中的肢体关节在各帧中的旋转数据来描述运动。BVH 文件已经成为了一种通用的人体特征动画文件格式,MotionBuilder、3DMax等动画制作软件皆支持bvh文件的导入和编辑。

Bvh文件由两部分组成,头部信息和之后的数据部分。Bvh文件头部信息的示例如图x所示,首先包含了目标物体的各个关节点及其继承关系,同时包括了描述关节点运动的旋转和位移两种通道,以及默认骨骼模型中各个关节点相对于父节点的局部偏移量。最后,头部信息中还包含了文件描述的动作的帧数以及每帧时长,默认单位为秒。在bvh文件的数据部分,从第一帧开始,以深度优先遍历骨骼树的顺序依次对每个关节点的各个数据通道进行记录,直到最后一帧。

更为具体地,图x中,HIERARCHY标记着关节继承信息描述的开始,ROOT为当前目标关节树唯一的根节点关节,JOINT指所有的子关节,OFFSET记录子关节相对于父关节的偏移量,CHANNELS表明数据记录部分对应的各个通道。特殊地,ROOT关节的通道中必须包含Xposition、Yposition、Zposition三个位移信息,而子节点的位移信息在具有rotation信息时为冗余数据,可以省略。

\subsection{人体骨骼模型}{}
综合考虑人体姿态信息的多样性和复杂性,为了最大限度表达人体姿态,同时考虑数据量因素,现该领域研究人员一般用人体各个关键点来估人体姿态信息。基于骨骼关键点方法首次由  Johansson 在其经典的移动灯显示实验提出。人体姿态的大部分信息由主要的关节关键点即可描述。由此,人体姿态估计领域的研究大多基于该描述方法。目前主流的数据集的标签也采用标注骨骼关键点的方式。例如上述Human3.6m数据集中的人体骨骼关节点定义如图x所示。在本文进行三维人体姿态估计时,输出的人体骨骼关节点定义如图x所示。以上人体骨骼模型也成为了由模型预测结果转为动作描述文件的基础。

由上一小节中对bvh文件的介绍可知,每个bvh文件描述一个运动目标,且运动目标由各个关节点构成,并以继承关系构建为骨骼树,具有唯一的父节点。故而对于本文所采集运动信息的人体目标,设计并搭建了如图x的人体骨骼架构,具体的继承关系如图x所示。

进一步地,基于人体解剖学知识,我们将各个子关节相对于父关节的偏移方向定义如下:

由此构成了人体骨骼模型的默认形态。在人体解剖学中,骨骼的运动由关节运动主导,关节的运动与关节面形状贴合,而关节面形状也是在人体的长期活动中,肌肉作用下逐步获得、形成的,以上击力使得人体关节运动一般为绕某个轴进行的旋转运动。在上述简化的人体骨骼模型下,每个骨骼的旋转平面皆可由其他骨骼的方向定义,这为构造bvh文件数据部分的欧拉角任务提供了清晰明确、易于定义的旋转轴。

\subsection{三维角色动作绑定}{}
在进行三维角色动画创作的过程中,当已经创建了一个骨骼结构的动作后,需要将动作与目标的实际模型进行捆绑,使得动作可以在目标模型上展示和播放,这个过程通常称为三维动画角色动作绑定或骨骼绑定(3D character rigging for animation)。

在本文的框架中,我们选择使用Autodesk的MotionBuilder软件,将上一小节生成的bvh文件与已有的人物模型进行绑定。


\section{实现效果}

\subsection{人体姿态估计}{}

\subsection{动作绑定}{}
