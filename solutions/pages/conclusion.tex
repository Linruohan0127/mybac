
\chapter{总结与展望}

\section{结论}{}

本文首先使用两步提升法对RGB单目单人视频分别进行了人体姿态序列的二维和三维的估计,并进一步搭建了一个完整的视频动作捕捉制作三维动画的框架。

% todo 可补充

在人体姿态估计任务方面,本文的工作可以对日常的人体姿态进行较为准确的还原,但目前的工作还具有以下局限性:
\begin{enumerate}
    \item 网络的泛化性不足。对快速的动作预测出的姿态速度较慢,对快速运动时不清晰帧的预测效果有限,对于非常见动作的预测准确性较差,在背景复杂时易出现混淆的情况。
    \item 网络对于上下文信息的提取能力有限。当对于多个三维姿态对应相同的二维姿态投影的歧义性发生时,易估计出反人体解剖学的三维姿态。
    \item 模型对于世界坐标系中人体根节点的位移的估计效果较差。
\end{enumerate}

在视频动作捕捉制作三维动画方面,本文的工作可以利用人体的动作制作出生动且还原度高的三维角色动画,但对于原始视频的采集仍具有以下要求:
\begin{enumerate}
    \item 固定相机位姿拍摄单人视频,保证人体部位全部入境。
    \item 目标人物距离与镜头的距离位于3到10米之间。
    \item 人物衣着简单,背景干净,避免遮挡与混淆。
\end{enumerate}

\section{展望}{}
针对如上问题,本文课题的未来工作仍然具有诸多挑战和极大的发展空间,也有诸多值得期待的改进方向:
\begin{enumerate}
    \item 在实际应用场景中,可以相应地丰富数据集,对不同距离,速度的动作进行全方面的采集。
    \item 人体姿态估计虽然具有着许多固有的难点,如投影歧义等。但人体本身的解剖学约束,以及人体动作中包含的丰富的语义信息以及上下文信息可以对姿态的估计提供高强度的帮助。可以利用如图卷积神经网络等模型,对如上深层信息进行进一步的提取。
    \item 视频动作捕捉的市场广阔,针对动作制作需求,可以对提取出的动作序列进行针对性的后处理优化,如消抖、平滑、清洗突变帧,稳定根节点位移,增强脚-地接触的重力效果等。
\end{enumerate}


