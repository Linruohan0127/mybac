
% 标题,中文
\ctitle{基于神经网络模型的人体姿态估计}

% 作者,中文
\cauthor{杜涵文}

% 学科,中文,本科生不需要
\csubject{}

% 学院,专业,班级,中文,本科毕业论文需要
\ccollege{电信}   % TODO 电信 or 计算机
\cmajor{计算机科学与技术(少年班)}
\cclass{计算机少71}

% 设计所在单位,中文,本科毕业论文需要
\corgnization{西安交通大学}

% 学号
\stunum{2150506076}

% 导师姓名,中文
\csupervisor{魏平}

% 关键词,中文。用全角分号「;」分割
% 研究生的应首先从《汉语主题词表》中摘选
\ckeywords{人体姿态识别;动作捕捉;卷积神经网络;半监督学习}

% 提交日期,本科生不需要
\cproddate{\the\year 年\the\month 月}

% 论文类型,中文,本科生不需要
% 从理论研究、应用基础、应用研究、研究报告、软件开发、设计报告、案例分析、调研报告、其它中选择
\ctype{}

% 论文标题,英文
\etitle{Human Pose Estimation Based on Deep Learning}

% 作者姓名,英文
\eauthor{Du Hanwen}
%todo 名字可能有错

% 学科,英文,本科生不需要
\esubject{}

% 导师姓名,英文
\esupervisor{Wei Ping}

% 关键词,英文。用半角分号和一个半角空格「; 」分割
\ekeywords{Human Pose Estimation; Motion Capture; Convolutional Neural Networks; Semi-Supervised Learning}

% 学科门类,英文
% 从Philosophy(哲学)、Economics(经济学)、Law(法学)、Education(教育学)、Arts(文学)、
%   Science(理学)、Engineering Science(工学)、Medicine(医学)、Management Science(管理学)中选择 
\ecate{}

% 提交日期,英文,本科生不需要
% 应当和 cproddate 保持一致
\eproddate{\monthname{\month}\ \the\year}

% 论文类型,英文,本科生不需要
% 从Theoretical Research(理论研究)、Application Fundamentals(应用基础)、Applied Research(应用研究)、
%   Research Report(研究报告)、Software Development(软件开发)、Design Report(设计报告)、
%   Case Study(案例分析)、Investigation Report(调研报告)、其它(Other)中选择
\etype{}

% 摘要,中文。段间空行
\cabstract{

}

% 摘要,英文。段间空行
\eabstract{

}
