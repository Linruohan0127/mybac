
\chapter{绪论}
\echapter{Preface}

视觉在是人类的重要感官之一,也是人类获取信息效率最高的感官,是人类认识和理解外界的重要途经,计算机视觉也相应成为了人类社会工业界的重要支柱技术之一。随着深度学习的广泛应用,计算机视觉在近年有了蓬勃的发展,人体姿态估计(human pose estimation, HPE)任务也被注入了新的活力,有了越来越多的科研成果。同时随着基于摄像头的计算机视觉解决方案的广泛铺开,人体姿态估计也逐渐贴合了越来越多的实际场景,在人机交互、游戏、虚拟现实、视频监控、运动分析、医疗辅助等领域有着广泛的应用,是计算机视觉领域的热门研究课题。本文基于单目摄像头的视频采集,利用深度神经网络,完成了视频序列中单人的人体姿态估计任务,并有效地利用提取出的动作制作了人体模型三维动画。

\section{人体姿态估计}
\esection{}

人体姿态估计(human pose estimation, HPE)任务已经发展了几十年,是计算机视觉及深度学习任务中十分有挑战性的课题,其目的是从给定的图像或视频中确定人体关键点的位置。当深度学习方案进入大家的视野后,随着其在目标检测、图像分类、语义分割等许多计算机视觉的任务中都表现了出了远超传统方案的较高性能,人体姿态估计任务也发展出更多样的基于深度学习的方法,并且借此取得了快速的进步,主要的发展在于设计具有强大的学习能力的网络,更丰富的数据集,以及更多的人体模型等。

人体姿态估计有多项不同维度的子任务分类。根据描述人体姿态的数据类型,可以分为基于骨架主要关节点的姿态估计和描述人体蒙皮的形态估计。根据获取图像及视频的方案,可以分为单目单视角估计与多目多视角估计。根据图像及视频中待估计目标的个数可以分为单人姿态估计与多人姿态估计。根据估计时所利用的上下文信息可以分为单帧图像估计与帧序列即视频估计。

其中最主要的场景为单目RGB图像及视频中人体关节点坐标的恢复,并根据输出的数据维度,也即描述人体姿态关节点的坐标维度,可以划分为二维人体姿态估计与三维人体姿态估计两大部分。二维人体姿态估计的目标是定位图像中每个关键点的(x,y)坐标,而三维人体姿态估计的目标是推断三维空间中每个关键点的(x,y,z)坐标。

单目人体姿态估计具有其独特的特征和挑战,其挑战主要在于三点:

\begin{enumerate}
    \item 其一,柔性体构型意味着复杂的相互依赖的关节和高自由度的肢体,这可能会导致出现自遮挡及其他罕见且复杂的姿态。
    \item 其二,是人体呈现的多样性,包括服装的多样及自身极为相似的部位。
    \item 其三,复杂的拍摄环境常导致前景的遮挡,加之多样的拍摄角度和画幅的不完整,导致目标与周围人的相似部位出现混淆。
\end{enumerate}


\subsection{二维人体姿态估计}{}

早期的人体姿态估计融入了许多特征的构建,比如描述运动信息的梯度方向直方图(HOG, Histogram of Oriented Gradient)和小边特征(Edgelet)等,但这些方法在实践中并不具备精确估计的能力。

随着深度神经网络的发展,深度学习方法则更能够从元数据重提取出更充分的特征,产生了优异的结果,并在很大程度上超过了传统方法。基于方法分类的二维人体姿态估计任务如图x。

单人姿态估计从预测关键点的方法上主要分为直接回归方法与热力图方法两类。前者利用输出特征映射直接回归关键点,如facebook于2018年搭建的DensePose。后者则首先生成heatmap (heatmap中的像素值表示关键点在该位置存在的概率),并根据heatmap预测关键点,如2016年的Hourglass。

而多人姿态估计的整体思路主要分为两类。第一种为自顶向下的方法,指的是首先从全图的范围上进行人体检测,然后切入到每个目标人体的范围,进行单人关键点估计,如2016年提出的AlphaPose。自底向上的方法的第一步则是定位图像中的所有关节点,第二步是将关节点向上分组为人体,2016年提出利局部关联域(PAF, Part Affinity Field)方法进行关键点组装的OpenPose。

\subsection{三维人体姿态估计}{}

三维人体姿态估计相对于二维人体姿态估计具有更多难点,其一是单目图像及视频信息本身具有的歧义性,即一个二维的姿态投影可以对应多种三维的姿态。其二是需要满足严格的设备和环境条件的MoCap 三维姿态数据集的采集较为困难,故而对网络的泛化性要求也更为严格。

基于深度学习的三维人体姿态估计方法可主要分为两大思路。第一种思路为两步式,先从图像中提取出二维关节点坐标,再将其从二维提升到三维坐标,如Facebook AI于2018年提出的VideoPose三维模型。另外,也有方法选择端到端式,从视频和图像中直接回归出三维坐标,如Li等于2014年利用卷积神经网络搭建的包括关节检测与坐标回归两支路模型。




\section{应用背景}
\esection{}
人体姿态估计在实际应用中,为视觉监控、行为分析、自动驾驶、电影和动画、虚拟现实、人机交互(HCI, Humancomputer interaction)、医疗辅助、自动驾驶、运动动作分析等多种场景提供了丰富的人体几何和运动信息。

\subsection{三维视频动作捕捉}{}

动作捕捉(MoCap, Motion Caption)是记录运动并在数字角色模型上创建运动的过程。动作捕捉可用于各种应用场景,例如运动,医疗应用,人体工程学,娱乐和机器人技术,以及三维角色动画,预可视化以及虚拟电影制作。更为具体地,当动作捕捉应用于游戏开发和电影制作中时,则主要指记录演员动作,来制作动画等具有视觉效果的媒体内容。

而这其中的主要动作捕捉技术主要依赖于硬件设备,从硬件设备的采集数据的原理上来说可以分为光学式、惯性式等。而除了一般动作捕捉系统中的穿戴式设备外、感应摄像头、符合光线要求的场地场所、相应软件以及基础的服务等对专业性的要求极高,价格不菲。根据相关调研报告,全球动作捕捉市场在2018年达到了1.632亿美元,并且预计到2026年,可以增长至2.617亿美元。同时预测复合年增长率在2019-2026年内为8.13%。

由于动作捕捉的价格昂贵,市场需求大,对动捕质量的要求随着动画产业的发展也在不断地增加。同时随着三维人体姿态估计的效果逐步提升,基于RGB摄像视频的动作捕捉成为了降低动画制作门槛的新的解决方案。


\subsection{国内外研究现状}{}
无硬件设备需求的视频人体动作捕捉解决方案提供商在近几年,于国内外如雨后春笋般崛起并在持续地发展。

如国内云舶提供的对单人视频进行捕捉的小K动捕,对人物的脚-地接触进行了优化,使其接触稳定,无垂直的浮动。此外,美国DeepMotion也提供了包含单人动作捕捉以及人物追踪等多种视频解决方案。同时,Ridical也提供了粒度细至手指关节的实时视频动作捕捉API。



\section{本文主要工作内容}
\esection{}

\subsection{本文工作}{}
针对人体姿态估计任务,本文的工作主要聚焦到了基于深度神经网络模型,对日常场景中,单目RGB摄像头采集的单人动作视频进行姿态估计,并利用估计出的人物动作制作三维角色动画。整体流程如图x所示,首先输入单人视频,利用残差网络搭建二维关节点估计模型,并利用一维空洞卷积提取视频中的时序信息,将二维关节点坐标提升为三维坐标,最后将三维坐标转换为描述动作文件的人体骨架欧拉角,进行三维角色绑定。

\subsection{论文结构安排}{}
本文共包含六个章节,各章节的具体内容概括如下。

第一章:绪论。在绪论中,首先介绍了人体姿态估计的两大主要领域,二维人体姿态估计和三维人体姿态估计任务的具体目标,以及当前工作的研究现状。说明了人体姿态估计课题的研究意义以及在各实际领域的应用价值,尤其是在动作捕捉领域,具有旷阔的市场和可期的前景。最后对本文的主要工作进行了简要的概括 ,并对本文的章节安排进行了介绍。

第二章:人体姿态识别相关基础。在第二章中,针对计算机视觉尤其是人体姿态估计中常用到的经典的深度学习模型及方法进行了介绍,包括卷积神经网络、残差网络等。本章所重点介绍的网络构成了第三章及第四章模型的基础架构。

第三章:基于反卷积神经网络的二维人体姿态估计。本章重点说明了用于二维人体姿态估计的模型,一个相对于之前的Hourglass和CPN结构都更为简单明了的网络结构。模型利用热力图对人体关节进行二维坐标的预测,并在网络中利用反卷积进行上采样,用以得到高分辨率的特征图。

第四章:基于空洞卷积神经网络的三维人体姿态序列估计。本章首先针对三维人体姿态估计任务的主要问题进行了分析,并采用了二步提升法,同时利用卷积神经网络处理视频中的时序信息,进行三维人体姿态的估计。并采取重投影法利用无标签数据进行半监督训练。

第五章:视频动作捕捉。在本章中,利用第三章及第四章的网络模型所预测出的人体姿态,对人物角色模型进行动画制作。搭建了人体骨骼模型,将关节点坐标转换为了动作描述文件bvh所需的欧拉角。并在MotionBuilder软件中对人物模型进行了动作绑定,制作出了还原度极高且稳定的三维动画。

第六章:总结与展望。本章归纳和总结了本文针对人体姿态估计任务所做的工作,以及针对视频动作捕捉进行的工程搭建。对人体姿态估计的后续研究方向以及视频动作捕捉的未来进行了更深层次的探讨和展望。

